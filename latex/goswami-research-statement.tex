\documentclass{article}
\usepackage{multicol}
\begin{document}
\noindent
\textbf{Introduction}\\
The role of human host microbiomes in neuropsychiatric diseases such as has been studied extensively in the literature, e.g. for reviews see \cite{goswami_role_2021}, \cite{hashimoto_emerging_2023}, and \cite{bonnechere}. Furthermore they have been implicated in a number of neuropsychiatric disorders, such as ADHD, in \cite{bull-larsen_potential_2019}, in increasing severity of autism spectrum disorders, ASD, in children in \cite{TOMOVA2015179}, in Alzheimer's Disease in the elderly, in \cite{yk_microbiota-gut-brain_2018} and \cite{escobar_influence_2022}
\\~\\
Of the different microbiomes in the human body, (e.g. gut, oral, skin, vaginal), the gut microbiome is the most extensively studied in relation to neuropsychiatric disorders, see \cite{sorboni_comprehensive_2022}. It is not only the most studied but also the microbiome where modern machine learning techniques have been most frequently and fruitfully applied. 
\\~\\
This leaves a research gap for other microbiomes of the human body. For instance both \cite{goswami_role_2021} and \cite{tao_relationship_2024} both note the dearth of literature with respect to the oral microbiome. 

We propose, therefore, to take some of the tools, especially machine learning tools, used in gut microbiome analysis and apply them for the analysis of the oral microbiome. We propose this for three reasons:

\begin{enumerate}
	\item The oral microbiome is yet under-studied with the tools applied to other micribiomes, as noted earlier \cite{goswami_role_2021},\cite{tao_relationship_2024}.
	\item An emerging argument exists for an oral-microbiome-brain axis, OMBA, similar to the gut-brain axis, \cite{bowland_oral-microbiome-brain_2022}, \cite{xi_coming_2024}, \cite{y_did_2020}. The consensus seems to be that this area still needs to be studied.
	\item Large, robust, and mature dataset exists for the microbiomes, such as the Human Oral Microbiome Dataset, and extended Human Oral Microbiome Dataset\cite{homd}, Cultivated Oral Bacterial Genome Reference \cite{li_catalog_2023}, and even some smaller datasets such as the UAE Healthy Future Study participants of 330 Emirati citizens \cite{noauthor_human_nodate}, and less specialized datasets such as FinnGen, \cite{noauthor_finngen_nodate}.
\end{enumerate}
\textbf{Machine Learning Methods} 
Our main source of inspiration will be methods already applied to the gut microbiome. 
\\~\\

\textbf{Objective of Research}
\\~\\
\textbf{Tentative Timeline}
\bibliographystyle{naturemag}
\bibliography{../references/goswami-proposal.bib}
\end{document}